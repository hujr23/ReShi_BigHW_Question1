
\documentclass[12pt,a4paper]{article}% 文档格式
\usepackage{ctex,hyperref}% 输出汉字
\usepackage{times}% 英文使用Times New Roman

\title{\fontsize{18pt}{27pt}\selectfont% 小四字号,1.5倍行距
	{\heiti% 黑体 
		关于热湿交换大作业的说明}}% 题目

\author{\fontsize{12pt}{18pt}\selectfont% 小四字号,1.5倍行距
	{\fangsong% 仿宋
		Jiarui Hu}\\% 标题栏脚注
	\fontsize{10.5pt}{15.75pt}\selectfont% 五号字号,1.5倍行距
	{\fangsong% 仿宋
		(清华大学未央书院)}}% 作者单位,“~”表示空格

\date{\today}% 日期(这里避免生成日期)

\usepackage{amsmath,amsfonts,amssymb,amsthm,amscd}% 为公式输入创造条件的宏包

\usepackage{graphicx}% 图片插入宏包
\usepackage{subfigure}% 并排子图
\usepackage{float}% 浮动环境,用于调整图片位置
\usepackage[export]{adjustbox}% 防止过宽的图片

\usepackage{caption}
\usepackage{graphicx}
\usepackage{float} 
\usepackage{subfigure}
\usepackage{subcaption}

\usepackage{tikz}
\usepackage[numbers]{natbib}
\usepackage{abstract}% 两栏文档,一栏摘要及关键字宏包
\renewcommand{\abstracttextfont}{\fangsong}% 摘要内容字体为仿宋
\renewcommand{\abstractname}{\textbf{摘\quad 要}}% 更改摘要二字的样式

\usepackage{xcolor}% 字体颜色宏包
\newcommand{\red}[1]{\textcolor[rgb]{1.00,0.00,0.00}{#1}}
\newcommand{\blue}[1]{\textcolor[rgb]{0.00,0.00,1.00}{#1}}
\newcommand{\green}[1]{\textcolor[rgb]{0.00,1.00,0.00}{#1}}
\newcommand{\darkblue}[1]
{\textcolor[rgb]{0.00,0.00,0.50}{#1}}
\newcommand{\darkgreen}[1]
{\textcolor[rgb]{0.00,0.37,0.00}{#1}}
\newcommand{\darkred}[1]{\textcolor[rgb]{0.60,0.00,0.00}{#1}}
\newcommand{\brown}[1]{\textcolor[rgb]{0.50,0.30,0.00}{#1}}
\newcommand{\purple}[1]{\textcolor[rgb]{0.50,0.00,0.50}{#1}}% 为使用方便而编辑的新指令

\usepackage{url}% 超链接
\usepackage{bm}% 加粗部分公式
\usepackage{multirow}
\usepackage{booktabs}
\usepackage{epstopdf}
\usepackage{epsfig}
\usepackage{longtable}% 长表格
\usepackage{supertabular}% 跨页表格
\usepackage{algorithm}
\usepackage{algorithmic}
\usepackage{changepage}% 换页

\usepackage{esint}%多重积分宏包
\usepackage{unicode-math}

\usepackage{enumerate}% 短编号
\usepackage{caption}% 设置标题
\captionsetup[figure]{name=\fontsize{10pt}{15pt}\selectfont Figure}% 设置图片编号头
\captionsetup[table]{name=\fontsize{10pt}{15pt}\selectfont Table}% 设置表格编号头

\usepackage{indentfirst}% 中文首行缩进
\usepackage[left=2.50cm,right=2.50cm,top=2.80cm,bottom=2.50cm]{geometry}% 页边距设置
\renewcommand{\baselinestretch}{1.5}% 定义行间距(1.5)

\usepackage{fancyhdr} %设置全文页眉、页脚的格式
\pagestyle{fancy}
\hypersetup{colorlinks=true,linkcolor=black}% 去除引用红框,改变颜色
\usepackage{makecell}


\usepackage{bibentry}
\usepackage{natbib}


\usepackage{amsmath}
\numberwithin{equation}{section}
\numberwithin{figure}{section}

\setlength{\belowcaptionskip}{1cm} %调整图片标题与下文距离


\begin{document}% 以下为正文内容
\setcounter{section}{-1}
	\maketitle% 产生标题,没有它无法显示标题

	\lhead{}% 页眉左边设为空
	\chead{}% 页眉中间设为空
	\rhead{}% 页眉右边设为空
	\lfoot{}% 页脚左边设为空
	\cfoot{\thepage}% 页脚中间显示页码
	\rfoot{}% 页脚右边设为空

\newpage
\section{参数定义及说明}

\begin{table}[htbp]
\centering
\caption{MATLAB中各参数定义及其含义-1}
\begin{tabular}{cll}
\toprule
参数 & 含义 & 单位\\
\midrule
$T_{a}$ & 环境温度 & 28$^{\circ}$C \\
$RH$ & 环境相对湿度 & 85\% \\
$\rho_{air}$ & 空气密度 & 1.18 kg/m$^{3}$ \\
$\mu_{air}$ & 空气动力粘度 & 1.85$\times10^{-5}$ kg/(m·s) \\
$\sigma$ &空气与水滴表面的张力系数 & 0.072 N/m \\
$C_{p,air}$ & 空气比热容 & 1005 J/(kg·K) \\
$C_d$&气流拖拽系数&0.0063\\
$k_a$ &空气热导率&0.0253 W/(m·K)\\
$k_w$ &水的热导率&0.606 W/(m·K)\\
$D_{aw}$ &空气中水蒸气扩散系数&2.6$\times10^{-5}$ m$^{2}$/s\\
$M_{w}$ &水的摩尔质量&18.01528$\times10^{-3}$ kg/mol\\
$D_0$ &水滴初始直径&0.1 mm\\
$R_0$ &水滴初始半径&0.05 mm\\
$R_{gas}$ &通用气体常数&8.314 J/(mol·K)\\
$T_{sur}$ & 翅片表面温度&$^{\circ}$C\\
$\rho_w$ &水的密度&997 kg/m$^{3}$\\
$g$&重力加速度&9.81 m/s$^{2}$\\
$P$ &环境大气压&101325 Pa\\
$P_{s}$ &环境中水蒸气分压 & 3214Pa \\
$P_{sat}$ &水滴表面饱和水蒸气压 &Pa\\
\bottomrule
\end{tabular}
\end{table}


\begin{table}[htbp]
\centering
\caption{MATLAB中各参数定义及其含义-2}
\begin{tabular}{cll}
\toprule
参数 & 含义 & 单位\\
\midrule
$C_{p,w}$ &水的比热容&4186 J/(kg·K)\\
$u$&空气来流速度&m/s\\
$t$ &时间&s\\
$X_w$ &水滴表面水蒸气的摩尔分数&-\\
$T_w$ &水滴温度&$^{\circ}$C\\
$D_w$ &水滴直径&m\\
$R_w$ &水滴半径&m\\
$h_{fg}$ &水的汽化潜热&J/kg\\
$h_c$ &水滴表面对流传热系数&W/(m$^{2}$·K)\\
$h_m$ &水滴表面对流传质系数&m/s\\
$\theta$ &接触角& rad\\
$L$ &水滴湿润长度&m\\
$R_c$ &接触半径&m\\
$A$ &水滴表面积&m$^{2}$\\
$A_p$ &水滴在流动方向的投影面积&m$^{2}$\\
$F_{\sigma}$ &表面张力&W\\
$F_d$ &气流拖拽力&W\\
$F_g$ &重力&W\\
$V$&水滴体积&m$^{3}$\\
$N_w$ &水的摩尔流量&mol/s\\
$q$ &单位面积上的传热通量&W/m$^{2}$\\
$Re$ &雷诺数&-\\
$Pr$ &普朗特数&0.707\\
$Sc$ &施密特数&-\\
$Nu$ &努塞尔数&-\\
$Sh$ &舍伍德数&-\\
$We$ &韦伯数&-\\
$r_{sl}$ &水滴与翅片导热热阻&K/W\\
\bottomrule
\end{tabular}
\end{table}


\newpage

\section{ODE函数的创建:变量关联}

\subsection{课件中给出的公式}
\noindent 表面张力:
\begin{equation}
F_{\sigma}=\sigma L \sin\theta
\end{equation}
气流拖拽力:
\begin{equation}
F_{d}=\frac{1}{2}C_{d}\rho_{air} A_{p} u^{2}
\end{equation}
接触角为$\theta$的球形帽状水滴体积:
\begin{equation}
V=\frac{\pi R^{3}}{3}(2-3\cos\theta+\cos^{3}\theta)
\end{equation}
其表面积$A$和在投影方向的面积$A_{p}$分别为:
\begin{equation}
A=2\pi R^{2}(1-\cos\theta)
\end{equation}
\begin{equation}
A_{p}=\pi R^{2}\sin^{2}\theta
\end{equation}
接触半径$R_c$与曲率半径的关系为:
\begin{equation}
R_{c}=R\sin\theta
\end{equation}
湿润长度$L$与接触半径的关系为:
\begin{equation}
L=2R\sin\theta 
\end{equation}

\subsection{无量纲参量}

对于水滴,其毕渥数$Bi$定义为:
\begin{equation}
Bi=\frac{h_{c} R}{k_{w}}
\end{equation}
其中$k_{w}$为水的导热系数。集总参数的应用条件为$Bi<0.1$,所以水滴在这个模型中可以认为是温度均匀的。


水滴表面水蒸气的浓度梯度是水蒸气扩散的主要动力,水滴表面的水蒸气浓度视为在此温度下的饱和水蒸气浓度。水滴表面的水
蒸气浓度摩尔通量计算如下:
\begin{equation}\label{eq:Nw}
	N_w=h_m (\frac{P_{sat}(T_w)}{R_{gas} T_w}-X_w\frac{P_{s}}{R_{gas} T_{a}})
\end{equation}
根据ASHRAE手册,饱和水蒸气压$P_{sat}$可由下式计算\footnote{ASHRAE. 2013 ASHRAE Handbook Fundamentals SI Edition [M].}:
\begin{equation}
\ln	P_{sat}(T_w)= c_1 T_w^{-1}+c_2 + c_3 T_w + c_4 T_w^{2} + c_5 T_w^{3} + c_6 \ln T_w
\end{equation}
其中各常数值为:
\begin{align*}
c_1 &=-5.8002206\times10^{3},\\
c_2 &=1.3914993\times10^{0},\\
c_3 &=-4.8640239\times10^{-2},\\
c_4 &=4.1764768\times10^{-5},\\
c_5 &=-1.4452093\times10^{-8},\\
c_6 &=6.5459673\times10^{0}.
\end{align*}


水滴质量在微元时间内变化可以表示为:
\begin{equation}\label{eq:dmw}
	m_w(t+\Delta t)=m_w(t)-N_w A_p M_w \Delta t
\end{equation}
水滴的温度变化可表示为:
\begin{equation}\label{eq:dTw}
	m_w C_{p,w} \frac{\mathrm{d}T_w}{\mathrm{d}t}=[h_c A (T_{a}-T_w)+\pi R_c^2 (T_{sur}-T_w)/r_{sl}]-\frac{\mathrm{d} m_w}{\mathrm{d} t} h_{fg}
\end{equation}
其中:
\begin{equation}
	h_{fg}=2500+1.84 T_w -4.19 T_w = 2500 -2.35 T_w \quad (kJ/kg)
\end{equation}
热阻$r_{sl}$取:
\begin{equation}
	r_{sl}=\frac{\delta_{eff}}{k_w \pi R_c^2}
\end{equation}
其中$\delta_{eff}$为等效液膜厚度,取$10^{-5}m$。
式\ref{eq:Nw}中的对流传质系数$h_m$可以通过宣乌特准则数计算关联式得到。表达式如
下:
\begin{equation}\label{eq:Sh}
	Sh=\frac{h_m D_w}{D_{aw}}=2+0.6 Re^{1/2} Sc^{1/3}
\end{equation}

施密特准则数Sc定义如下:
\begin{equation}
	Sc=\frac{C_{p,air} \mu_{air}}{k_a}
\end{equation}
式\ref{eq:dTw}中的对流传热系数$h_c$可以通过计算努谢尔特关联式得到,计算表达式如下:
\begin{equation}\label{eq:Nu}
	Nu =\frac{h_c D_w}{k_a}=2+0.6 Re^{1/2} Pr^{1/3}
\end{equation}
式\ref{eq:Nu}和式\ref{eq:Sh}中的雷诺数$Re$定义如下:
\begin{equation}
	Re=\frac{\rho_{air} u D_w}{\mu_{air}}
\end{equation}
关于韦伯数$We$,其定义如下:
\begin{equation}
	We=\frac{\rho_{air} u^{2} D_w}{2\sigma}
\end{equation}
\subsection{接触角理论}
本部分中,我们假定表冷器翅片材料为铝材,静态接触角$\theta_s$依靠托马斯-杨的润湿理论:
\begin{equation}
	\cos\theta_s=\frac{\sigma_{sv}-\sigma_{sl}}{\sigma_{lv}}
\end{equation}
其中$\sigma_{sv}$为固-气界面张力,$\sigma_{sl}$为固-液界面张力,$\sigma_{lv}$为液-气界面张力。
在我们的研究案例中,这个$\theta_s$取$\pi/5$且变化极小,可以认为是常数。
但当液滴上方存在空气流动时,接触角会发生变化,这时需要用到动态接触角理论,我们
采用Cox-Voinov模型\footnote{Chan TS, Kamal C, Snoeijer JH, Sprittles JE, Eggers J. Cox–Voinov theory with slip. Journal of Fluid Mechanics. 2020;900:A8. doi:10.1017/jfm.2020.499}
来描述动态接触角与静态接触角的关系:
\begin{equation}
	\theta^{3}=\theta_{s}^{3}+\frac{9 \mu_{air} u}{\sigma} \ln(\frac{L}{L_{m}})
\end{equation}
其中$\theta$为动态接触角,$\theta_{s}$为静态接触角,$\mu_{air}$为空气动力粘度,$u$为空气来流速度,$\sigma$为空气与水滴表面的张力系数,$L$为水滴湿润长度,$L_{m}$为分子尺度,一般取10$^{-9}$ m。
这是我们实际要用到的接触角,它是关于来流速度$u$和半径$R$的函数。



\subsection{力学平衡关系:水滴的生长、脱落、滑移与破碎}
首先来考虑水滴的生长$(\mathrm{d}R/\mathrm{d}t)$。根据\ref{eq:dmw},水滴的体积变化率为:
\begin{equation}
	\frac{\mathrm{d}V}{\mathrm{d}t}=\frac{1}{\rho_w}\frac{\mathrm{d}m_w}{\mathrm{d}t}=-\frac{N_w A_p M_w}{\rho_w}
\end{equation}
也就是说,对于一个固定的来流速度$u$,水滴生长的速率$\mathrm{d}R/\mathrm{d}t$可以通过下式计算得到:
\begin{equation}
	-N_w \pi R^2 \sin^{2}\theta M_w =\rho_w \pi R^{2}(2-3\cos\theta+\cos^{3}\theta) \frac{\mathrm{d}R}{\mathrm{d}t}
\end{equation}
简化得:
\begin{equation}
	\frac{\mathrm{d}R}{\mathrm{d}t}=-\frac{N_w M_w}{\rho_w (2-3\cos\theta+\cos^{3}\theta)\sin^{2}\theta}
\end{equation}
所以水滴的生长速率用一个偏微分方程表示,只与来流速度$u$和半径$R$有关。因为$\theta=\theta(u,R)$。


对于贴附在水平上壁面的水滴,其所受重力$F_g$为:
\begin{equation}
	F_g=\rho_w g\cdot \frac{1}{3}\pi R^3 (2-3\cos\theta+\cos^{3}\theta)
\end{equation}

表面张力提供的粘附力$F_{\sigma}$用于抵抗脱落力,
当水滴所受重力大于粘附力时,水滴脱落。水滴脱落的条件为:
\begin{equation}
	F_g > F_{\sigma}
\end{equation}


水滴滑移与气流拖拽力$F_d$有关,当气流拖拽力大于粘附力时,
水滴开始滑移。根据牛顿的粘附力模型,水滴滑移的临界条件为:
\begin{equation}
	F_d > \mu_{air}A_p\frac{\partial u}{\partial y}|_{y=0} \approx \mu_{air} A_p \frac{u}{R_c}
\end{equation}
也就是:
\begin{equation}
	R\cdot (u\sin\theta)=\frac{2\mu_{air}}{C_d\rho_{air}}
\end{equation}








最后来讨论水滴的破碎。当韦伯数$We$大于临界韦伯数$We_{dro}$时,水滴破碎。
本文采用 SSD(Stochastic Secondary Droplet)模型模拟液滴在湿通道中的破碎过程。
在气液两相流动过程中,大液滴在气流的冲击下会破碎成小液滴。SSD 模型允许液滴在
一定范围内随机破碎。在破碎过程中,小液滴的尺寸具有一定独立性。小液滴的粒径符
合 Fokker-Planck 分布。区分液滴大小之间的临界尺寸使用下面的公式来计算:
\begin{equation}
	r_{dro}=\frac{We_{dro} \cdot \sigma}{\rho_{air} u^{2}}
\end{equation}
关于临界韦伯数$We_{dro}$,本文采用源流流动中常出现的经验值10。

大液滴破碎后,小液滴产生,小水滴的直径可以采用下式计算:
\begin{equation}
	T(x)=\frac{1}{\sqrt{2\pi\xi^2}}\exp[\frac{-(x-x_0-\xi)^2}{2\xi^2}]
\end{equation}
其中x为水滴破碎后产生的小水滴的半径的对数分布形式:
\begin{equation}
	x=\ln(r)
\end{equation}
$\xi^2$是新生成的水滴的半径方差的一个对数表达形式:
\begin{equation}
	\xi^2=-0.1\ln (We/We_{dro})
\end{equation}

最后我们从\ref{eq:dTw}中提取以翅片表面温度$T_{sur}$为自变量的水滴温度变化率:
\begin{equation}
	\frac{\mathrm{d}T_w}{\mathrm{d}t}=\frac{h_c A (T_{a}-T_w)+\pi R_c^2 (T_{sur}-T_w)/r_{sl}-\frac{\mathrm{d} m_w}{\mathrm{d} t} h_{fg}}{m_w C_{p,w}}
\end{equation}
上式含自变量$T_{sur},R,u$。
\subsection{总结}
考虑时间$t$,由前面的推导可知,水滴的半径$R$似乎是一个隐函数(或偏微分方程)的形式,可以表示为:
\begin{gather}
	R=R(u,t)\\
	\frac{\mathrm{d}R}{\mathrm{d}t}=-\frac{N_w M_w}{\rho_w (2-3\cos\theta+\cos^{3}\theta)\sin^{2}\theta}\\
	h_m=h_m(u,R(u,t))=h_m(u,t)\\
	\theta=\theta(u,R(u,t))=\theta(u,t)\\
	R(u,t=0)=R_0=0.05 mm
\end{gather}

对于水的温度$T_w$,其也可以表示为一个隐函数的形式,自变量含有$u,t,T_{sur}$,可以表示为:
\begin{gather}
	T_w=T_w(u,t,T_{sur},R(u,t))=T_w(u,t,T_{sur})\\
	\frac{\mathrm{d}T_w}{\mathrm{d}t}=\frac{h_c A (T_{a}-T_w)+\pi R_c^2 (T_{sur}-T_w)/r_{sl}-N_w A_p M_w h_{fg}}{m_w C_{p,w}}\\
	h_c=h_c(u,R(u,t))=h_c(u,t)\\
	R(u,t=0)=R_0=0.05 mm\\
	h_{fg}=h_{fg}(T_w)\\
	m_w=m_w(R(u,t),\theta(u,t))=m_w(u,t)																																																																																																														
\end{gather}																																																																																																														
																																																																																																															
																																																																																																															
																																																																																																											
																																																																																																															
																																																																																																															
																																																																																																															
																																																																																																															
																																																																																																															
																																																																																																															
																																																																																																															
																																																																																																															
																																																																																																															
																																																																																																															
																																																																																																															
																																																																																																															
																																																																																																															
																																																																																																															
																																																																																																															
																																																																																																															
																																																																																																															
																																																																																																															
																																																																																																															
																																																																																																															
																																																																																																															
																																																																																																															
																																																																																																															
																																																																																																															
																																																																																																															
																																																																																																															
																																																																																																															
																																																																																																															
																																																																																																															
																																																																																																															
																																																																																																															
																																																																																																															
																																																																																																															
																																																																																																															
																																																																																																															
																																																																																																															
																																																																																																															
																																																																																																															
																																																																																																															
																																																																																																															
																																																																																																															
																																																																																																															
																																																																																																															
																																																																																																															
																																																																																																															
																																																																																																															












\end{document}% 结束